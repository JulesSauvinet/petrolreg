\documentclass[12pt]{article}
\usepackage[french]{babel}
\usepackage[utf8]{inputenc}
\usepackage{amsmath}
\usepackage{graphicx}
\usepackage[colorinlistoftodos]{todonotes}
\usepackage{float}
\usepackage{graphicx}
\usepackage{caption}
\usepackage{subcaption}
\usepackage{adjustbox}
\usepackage[euler]{textgreek}
\usepackage[left=1in,right=1in,top=1in,bottom=1in]{geometry}
%\usepackage{subfigure}
\usepackage{comment}
\usepackage{caption}
\usepackage{lastpage}
\usepackage[colorlinks,pdfpagelabels,pdfstartview = FitH,bookmarksopen = true,bookmarksnumbered = true,linkcolor = black,plainpages = false,hypertexnames = true,citecolor = black,pagebackref = true,urlcolor = black] {hyperref}
\usepackage{setspace}
\usepackage{silence}
\WarningFilter{latex}{Text page}
\usepackage{parskip}
\newcommand{\cms}{~cm\textsuperscript{-2}s\textsuperscript{-1} }

\graphicspath{{figures/}}   
% \renewcommand{\figurename}{Fig.}
\addto\captionsenglish{\renewcommand{\figurename}{Fig.}}

\begin{document}

\begin{titlepage}

\newcommand{\HRule}{\rule{\linewidth}{0.5mm}} % Defines a new command for the horizontal lines, change thickness here

\center % Center everything on the page
 
%----------------------------------------------------------------------------------------
%	HEADING SECTIONS
%----------------------------------------------------------------------------------------

\textsc{\LARGE Universit\'e Claude Bernard Lyon I}\\[1.5cm] % Name of your university/college
\textsc{\Large Master Data Science}\\[0.5cm] % Major heading such as course name
\textsc{\large Mod\`eles de r\'egression}\\[0.5cm] % Minor heading such as course title

%----------------------------------------------------------------------------------------
%	TITLE SECTION
%----------------------------------------------------------------------------------------

\HRule \\[0.4cm]
{ \huge \bfseries R\'egression des donn\'ees d'extraction de puits de p\'etrole au Canada}\\[0.4cm] % Title of your document
\HRule \\[1.5cm]
 
%----------------------------------------------------------------------------------------
%	AUTHOR SECTION
%----------------------------------------------------------------------------------------

\begin{minipage}{0.4\textwidth}
\begin{flushleft} \large
\emph{Auteurs:}\\
Bruno \textsc{Dumas}\\% Your name
Jules \textsc{Sauvinet}
\end{flushleft}
\end{minipage}
~
\begin{minipage}{0.4\textwidth}
\begin{flushright} \large
\emph{Professeur:} \\
Fran\c cois \textsc{Wahl} % Supervisor's Name
\end{flushright}
\end{minipage}\\[2cm]

% If you don't want a supervisor, uncomment the two lines below and remove the section above
%\Large \emph{Author:}\\
%John \textsc{Smith}\\[3cm] % Your name

%----------------------------------------------------------------------------------------
%	DATE SECTION
%----------------------------------------------------------------------------------------

{\large \today}\\[1cm] % Date, change the \today to a set date if you want to be precise

%----------------------------------------------------------------------------------------
%	LOGO SECTION
%----------------------------------------------------------------------------------------

\includegraphics[height=5cm]{ucbl}\\[1cm] % Include a department/university logo - this will require the graphicx package
 
%----------------------------------------------------------------------------------------

\vfill % Fill the rest of the page with whitespace

\end{titlepage}

\begin{abstract}
L'objectif est d'effectuer des r\'egressions des valeurs d'extractions des puits de p\'etrole du Canada afin de pouvoir pr\'edire les futures donn\'ees d'extraction. 
Les courbes des valeurs des donn\'ees d'extraction donnent une classification de la qualit\'e des puits. L'\'etiquetage de la qualit\'e de chaque puits \`a \'et\'e effectu\'e par des experts. 
L'objectif de ce travail est de mettre en place une classification automatique de ces puits: La
d\'emarche propos\'ee est d\'ecrite ci-dessous.
L'id\'ee est de remplacer ces courbes par des fonctions param\'etriques.
Pour cela, plusieurs r\'egressions vont \^etre envisag\'ees afin d'avoir la meilleure pr\'ediction/classification possible.
\end{abstract}

%\begin{spacing}{1.15}
\pdfbookmark[1]{Contents}{toc}
\small{
\tableofcontents 
}


\newpage
% \listoffigures
% \newpage
% \listoftables 
%\end{spacing}


\section{R\'egressions polynomiales}
\label{sec:reg_pol}

Une fa\c con simple est d'ajuster un polyn\^ome de degr\'e faible sur chacune des courbes et
de voir si les coefficients pr\'esentent des clusters, c'est \`a dire des groupes de points distincts
quand on les regarde dans l'espace.
On essaiera des polyn\^omes de degr\'e $0$, $1$, $2$, $3$, et $4$.
On pr\'esentera les courbes de production simul\'ees obtenues, comme dans la figure ci-dessous.
 
\begin{figure}[H]
 \centering % avoid the use of \begin{center}...\end{center} and use \centering instead (more compact)
	\includegraphics[width=400px]{reg_pol}
  \caption{\label{fig:polynomial_regressions} Courbes non r\'egress\'ees, et courbes de r\'egressions polynomiales de degr\'es $0$,$1$,$2$,$3$ et $4$}
\end{figure}

La moyenne des R-Squared ajusted pour les $75$ courbes pour les $5$ types de r\'egression sont respectivement : 
\newline
$0.0000000$, $0.4853609$, $0.6622269$, $0.7202006$ et $0.7557633$.
\newline
~\\
On voit que plus le degr\'e du polynome augmente, plus la r\'egression est de qualit\'e au crit\`ere du R-Squared (les valeurs pr\'edites se rapprochent des valeurs des observations).

Si on se r\'ef\`ere \`a la figure ci-dessus, on observe que la plupart des simulations
pr\'esente une remont\'ee au bout de quelques mois, que certaines simulations sont concaves
au lieu d'\^etre convexes, voire que certaines d'entre elles pourraient avoir des valeurs n\'egatives (autour des $35$ mois d'exploitation).

\newpage


\section{R\'egressions exponentielles}

Une id\'ee simple pour corriger ces d\'efauts est d'utiliser une autre forme param\'etrique pour
les simulations. Une suggestion imm\'ediate pour qui a un peu l'habitude de ces courbes est
une forme exponentielle du style : $y=k_{0}.e^{-k_{1}.t}$ où $y$ est la production, $t$ le mois, et $k0$ et
$k1$ deux param\`etres \`a d\'eterminer.

\textbf{R\'egression exponentielle $log(y)=ax+b$}

\begin{figure}[H]
 \centering % avoid the use of \begin{center}...\end{center} and use \centering instead (more compact)
	\includegraphics[width=290px]{reg_exp_1}
  \caption{\label{fig:exponential_reg_lm} Courbes ajust\'ees avec une fonction exponentielle et l'outil lm de R}
\end{figure}


\textbf{R\'egression exponentielle $y=k_{0}.e^{-k_{1}.x}$}

\begin{figure}[H]
 \centering % avoid the use of \begin{center}...\end{center} and use \centering instead (more compact)
	\includegraphics[width=290px]{reg_exp_2}
  \caption{\label{fig:exponential_reg_nls} Courbes ajust\'ees avec une fonction exponentielle et l'outil nls de R}
\end{figure}

La r\'egression $log(y)=ax+b$ faite avec lm est plus fidèle aux données, on s'int\'eressera donc \`a celle-ci.
\newline 

La moyenne du "ajusted R-Squared" pour les $75$ r\'egressions exponentielles est cette fois-ci de $0.6754964$, donc de moins bonne quali\'e \`a priori que les r\'egressions polynomiales de degr\'e sup\'erieures ou \'egale \`a $3$. 
\newline
N\'eanmoins, quand on s'int\'eresse au d\'etail des R-Squared, on trouve de tr\`es bonne r\'egressions avec un R-Squared tr\`es bon comme des r\'egressions avec un R-Squared tr\`es mauvais. Cela est probablement d\^u à certaines valeurs "outliers" qui ne permettent pas \`a la fonction exponentielle de s'adapter aux courbes. 
\newline
La partie $5$ et le lissage des courbes permettra de pallier ce probl\`eme et probablement de montrer que la r\'egression exponentielle est adapt\'ee si un travail d'att\'enuation des "pics" est fait au pr\'ealable.
\newline

~\\
$10$ premiers R-Squared : 
\newline
$0.8011383$, $0.4182548$, $0.7403922$, $0.7585257$, $0.4186529$, $0.6888943$, $0.3297291$, $0.818889$, $0.7745934$ et $0.7527809$.
\newline

On peut d'ores et déjà constater que les trois classes de puits définies par les experts sont visuellement distinguables, bien qu'un peu entremêlées. Ceci indique qu'il sera possible de mettre en place un modèle de classification des puits.
\newline

\textbf{Autres possibilit\'es}
\newline
Nous avons pens\'e a un ajustement par fonction inverse du type $y = \frac{a}{x} + b$ avec $y$ la production, $x$ le mois et $a$ et $b$ des constantes à déterminer. On peut aussi penser à une gaussienne, dont l'allure s'approche de celles des courbes.



\newpage

\section{Courbes hautes et basses \`a 95\%}

Quelles sont les incertitudes sur les r\'egressions des points 2 ? Plus concr\`etement, on
vous demande de tracer pour un exemple de chaque type de courbe, la courbe haute (\`a
95\%) et la courbe basse (toujours \`a 95\%)

\textbf{Intervalles de confiance avec nls}

\begin{figure}[H]
 \centering % avoid the use of \begin{center}...\end{center} and use \centering instead (more compact)
	\includegraphics[width=430px]{q3_predict_nls}
  \caption{\label{fig:q3_predict_nls} Courbes hautes et basses à 95\% pour un exemple de chaque classe de courbes}
\end{figure}

\textbf{Intervalles de confiance avec lm}

\begin{figure}[H]
 \centering % avoid the use of \begin{center}...\end{center} and use \centering instead (more compact)
	\includegraphics[width=430px]{q3_predict_lm}
  \caption{\label{fig:q3_predict_lm} Courbes hautes et basses à 95\% pour un exemple de chaque classe de courbes}
\end{figure}

Comme le montre les figures ci-dessus, il y a davantage d'incertitude sur les valeurs des courbes de qualit\'e 'bad' et 'm\'edium'. Cela induit que les valeurs de ces courbes suivent des tendances plus compliqu\'ees à ajuster. En effet, les valeurs d'extraction pour ces puits sont probablement plus impr\'evisible et suivent parfois quelques oscillations. En outre, les incertitudes aux valeurs d'extraction pour les premiers et derniers mois sont \'egalement plus fortes. 


\newpage
\section{Reclassement avec r\'egression logistique}
 
En examinant le graphe k1 fonction de k0, on se rend compte que certaines courbes
class\'ees 'Good' par les experts donnent l'impression d'\^etre plut\^ot 'medium', tandis que
certaines 'bad' pourraient \^etre aussi 'medium'. 
\newline
Avez-vous des suggestions sur 5 courbes au
plus qui pourraient \^etre mal class\'ees ? 
\newline
Justifiez vos choix (i.e. une fa\c con de faire est d'effectuer une r\'egression logistique dont le y est la classe pr\'edite par l'expert et les $x$ sont les coefficients $k0$ et $k1$, et d'examiner comment la r\'egression est am\'elior\'ee en changeant la
classe d'un point).

\textbf{Clustering des courbes en fonction de $k0$ et $k1$ avant reclassement}

\begin{figure}[H]
 \centering % avoid the use of \begin{center}...\end{center} and use \centering instead (more compact)
	\includegraphics[width=300px]{clustering}
  \caption{\label{fig:k0_k1} Coefficients $k1$ en fonction de $k0$ et classes des courbes}
\end{figure}

La couleur du cercle ext\'erieur est la classe donnée par les experts et le cercle int\'erieur est sa classe pr\'edite par le classifieur.
\newline 
$16$ courbes sont mal class\'ees d'apr\`es les pr\'edictions et l'AIC de la r\'egression est de $74.63465$. 
\newline 
On essaye d'améliorer le taux du classifieur en modifiant la classe prédite par les experts sur certaines courbes.
\newline 

On change ainsi la classe de $5$ puits. 
\newline
On se limite au reclassement de $5$ puits car on ne veut pas trop modifier le mod\`ele de d\'epart et laisser de la souplesse au classifieur si celui-ci doit traiter la classification de nouveaux puits à l'avenir.
\newline

On procède de manière itérative. On choisit un puits qui semble très éloigné du reste de sa classe et on change sa classe donnée par les experts pour celle prédite par le classifieur. On relance le modèle du classifieur sur cette nouvelle base de données. On compte les points mal classés avec ce nouveau classifieur. Si le puits choisi ne diminue pas le nombre de points bien classés, on lui réattribue sa valeur originelle et on choisit un autre point. Quand on troive un point qui convient, on répète ces opérations jusqu'à atteindre 5 points.
\newline

On aurait pu automatiser cette opération qui est un problème d'optimisation simple. Ce problème peut se résumer par la phrase suivante :
\newline
\textit{Trouver la séquence de 5 puits qui minimise le nombre de puits mal classés}
\newline

Apr\'es interpr\'etations graphiques et des tests de pr\'ediction de classe avec ou sans changement de classe des $16$ puits mal class\'es, on d\'etermine les $5$ puits qui r\'eajustent au mieux le mod\'ele et am\'eliorent le classifieur :
\newline
Les courbes des puits '$Well-288$' de good à médium, '$Well-333$' de bad à médium, '$Well-246$' de médium à bad, '$Well-257$' de good à médium, et '$Well-258$' de bad à médium.

\textbf{Clustering des courbes en fonction de $k0$ et $k1$ apr\`es reclassement}

\begin{figure}[H]
 \centering % avoid the use of \begin{center}...\end{center} and use \centering instead (more compact)
	\includegraphics[width=250px]{clustering2}
  \caption{\label{fig:k0_k1} Coefficients $k1$ en fonction de $k0$ et classes des courbes}
\end{figure}

L'AIC de la r\'egression est de $52.52519$ et il n'y a plus que $10$ courbes mal class\'ees pour $5$ de chang\'ees sur les $16$ de d\'epart. Soit $1$ courbe qui dont la pr\'ediction de classe par le classifieur s'est accord\'ee sur sa vraie classe. 
\newline

A présent que nous avons établi un classifieur, nous pouvons nous demander s'il est possible d'en améliorer la précision. On peut par exemple tenter de produire un meilleur modèle sur les données en lissant les courbes avant d'effectuer la régression.

\newpage

\section{Gestion des spikes et lissage des courbes}

\textbf{R\'egression polynomiale de degr\'e $3$ avec smooth}

\begin{figure}[H]
 \centering % avoid the use of \begin{center}...\end{center} and use \centering instead (more compact)
	\includegraphics[width=250px]{reg_pol3_smooth_loess}
  \caption{\label{fig:reg_pol3_smooth_loess} R\'egression polynomiale de degr\'e 3 avec courbes liss\'ees au pr\'ealable avec loess}
\end{figure}

On obtient une moyenne de R-Squared de $0.9802724$ cette fois-ci apr\`es lissage des courbes. La r\'egression polynomiale devient ainsi tr\`es performante. Il reste \`a savoir si l'on peut se permettre l'approximation faite par le lissage des courbes et si les valeurs induites par les spikes avaient une pertinence intransigible. 
\newline
Toutefois, la r\'egression avec smooth ne r\`egle pas les valeurs qui remontent, les simulations concaves au lieu d'\^etre convexes, et les potentielles valeurs n\'egatives autour des $35$ mois.

\textbf{R\'egression exponentielle avec lissage des spikes}

\begin{figure}[H]
 \centering % avoid the use of \begin{center}...\end{center} and use \centering instead (more compact)
	\includegraphics[width=250px]{reg_exp_smooth_loess}
  \caption{\label{fig:reg_exp_smooth_loess} R\'egression exponentielle lm avec courbes liss\'ee au pr\'ealable avec loess}
\end{figure}

On obtient une moyenne de R-Squared de $0.7569813$ cette fois-ci apr\`es lissage des courbes. La r\'egression est sensiblement am\'elior\'ee avec le lissage des courbes mais reste moins efficace que la r\'egression polynomiale de degr\'e 3.

\newpage

\textbf{Reclassemement apr\`es lissage des courbes et ajustement exponentiel}

On effectue à nouveau une r\'egression logistique à partir des courbes liss\'ees comme à la question $4$.
On obtient cette toujours $16$ courbes mal class\'ees, un AIC de $85$ et le graphique suivant : 

\begin{figure}[H]
 \centering % avoid the use of \begin{center}...\end{center} and use \centering instead (more compact)
	\includegraphics[width=300px]{smooth_clustering_exp}
  \caption{\label{fig:reg_exp_clust} Clustering sur mod\`ele exponentiel avec courbes liss\'ees}
\end{figure}

On se limite comme dans la partie $3$ \'a seulement $5$ reclassement de courbes.
Apr\'es reclassement des puits  '$Well-290$', '$Well-333$', '$Well-312$', '$Well-288$', '$Well-258$', on obtient un AIC de $56$ et plus que $8$ courbes mal class\'ees en faisant une nouvelle r\'egression logistique, soit $3$ courbes dont la classe pr\'edite a cette fois-ci \'et\'e \'egale \'a celle observ\'ee.

\begin{figure}[H]
 \centering % avoid the use of \begin{center}...\end{center} and use \centering instead (more compact)
	\includegraphics[width=300px]{smooth_clustering_exp_2}
  \caption{\label{fig:reg_exp_clust_2} Clustering sur mod\`ele exponentiel avec courbes liss\'ee apr\'es reclassement}
\end{figure}

\newpage

\textbf{Reclassemement apr\`es lissage des courbes et ajustement polynomial}

On effectue une r\'egression logistique \'a partir des courbes liss\'ees et ajuster par un polyn\^ome de degr\'e $3$ cette fois-ci.
On obtient cette toujours $12$ courbes mal class\'ees, un AIC de $78.25367$ et le graphique de dépendance des $4$ coefficients entre eux suivant : 

\begin{figure}[H]
 \centering % avoid the use of \begin{center}...\end{center} and use \centering instead (more compact)
	\includegraphics[width=290px]{cluster_poly_1}
  \caption{\label{fig:reg_exp_clust} Clustering sur mod\`ele polynomial de degr\'e $3$ avec courbes liss\'ees}
\end{figure}


Apr\'es reclassement des puits  '$Well-257$', '$Well-250$', '$Well-333$', '$Well-266$' on obtient un AIC de $64.16384$ et plus que $6$ courbes mal class\'ees en faisant une nouvelle r\'egression logistique, soit $2$ courbes dont la classe pr\'edite a cette fois-ci \'et\'e \'egale \'a celle observ\'ee.

\begin{figure}[H]
 \centering % avoid the use of \begin{center}...\end{center} and use \centering instead (more compact)
	\includegraphics[width=290px]{cluster_poly_2}
  \caption{\label{fig:reg_exp_clust} Clustering sur mod\`ele polynomial de degr\'e $3$ avec courbes liss\'ees et reclassement}
\end{figure}

\section*{Acknowledgments}
Nos remerciements vont à Fran\c cois Wahl pour l'enseignement de son cours "Mod\`eles de r\'egression" à l'Universit\'e Claude Bernard Lyon I et son accompagnement durant ce projet.


\appendix
\section{Code de la question 1}
\begin{figure}[H]
 \centering % avoid the use of \begin{center}...\end{center} and use \centering instead (more compact)
	\includegraphics[width=450px]{code_q1}
  \caption{\label{fig:code_q1} Code de la question 1 des r\'egressions polynomiales}
\end{figure}

\section{Code de la question 2}
\begin{figure}[H]
 \centering % avoid the use of \begin{center}...\end{center} and use \centering instead (more compact)
	\includegraphics[width=600px]{code_q2_1}
  \caption{\label{fig:code_q2_1} Code de la question 2 des r\'egressions exponentielles, partie 1}
\end{figure}
\begin{figure}[H]
 \centering % avoid the use of \begin{center}...\end{center} and use \centering instead (more compact)
	\includegraphics[width=600px]{code_q2_2}
  \caption{\label{fig:code_q2_2} Code de la question 2 des r\'egressions exponentielles, partie 2}
\end{figure}

\section{Code de la question 3}
\begin{figure}[H]
 \centering % avoid the use of \begin{center}...\end{center} and use \centering instead (more compact)
	\includegraphics[width=600px]{code_q3_1}
  \caption{\label{fig:code_q3_1} Code de la question 3 des intervalles de confiance, partie 1}
\end{figure}
\begin{figure}[H]
 \centering % avoid the use of \begin{center}...\end{center} and use \centering instead (more compact)
	\includegraphics[width=600px]{code_q3_2}
  \caption{\label{fig:code_q3_2} Code de la question 3 des intervalles de confiance, partie 2}
\end{figure}

\section{Code de la question 4}
\begin{figure}[H]
 \centering % avoid the use of \begin{center}...\end{center} and use \centering instead (more compact)
	\includegraphics[width=600px]{code_q4_1}
  \caption{\label{fig:code_q4_1} Code de la question 4 des reclassements de courbes apr\`es r\'egression logistique, partie 1}
\end{figure}
\begin{figure}[H]
 \centering % avoid the use of \begin{center}...\end{center} and use \centering instead (more compact)
	\includegraphics[width=600px]{code_q4_2}
  \caption{\label{fig:code_q4_2} Code de la question 4 des reclassements de courbes apr\`es r\'egression logistique, partie 2}
\end{figure}
\section{Code de la question 5}
\begin{figure}[H]
 \centering % avoid the use of \begin{center}...\end{center} and use \centering instead (more compact)
	\includegraphics[width=600px]{code_q5_1}
  \caption{\label{fig:code_q5_1} Code de la question 5 des lissages de spikes puis des reclassements de courbes, partie 1}
\end{figure}
\begin{figure}[H]
 \centering % avoid the use of \begin{center}...\end{center} and use \centering instead (more compact)
	\includegraphics[width=600px]{code_q5_2}
  \caption{\label{fig:code_q5_2} Code de la question 5 des lissages de spikes puis des reclassements de courbes, partie 2}
\end{figure}

\end{document}
